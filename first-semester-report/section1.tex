%%
%%  Department of Electrical, Electronic and Computer Engineering.
%%  EPR400/2 First Semester Report - Section 1.
%%  Copyright © 2011-2021 University of Pretoria.
%%

\section{Literature study}

The use of artificial systems to emulate tasks that humans find straightforward to perform, such as solving 2D puzzles, has been a long-standing practice since components of the solution system are often relevant in industrial applications \cite{Burdea:Solving_Jigsaw_Puzzles_by_a_Robot}. This work focuses on a similar task that involves the construction of 3D shapes using small cubes. Such a task bears similarity to those tasks in the domain of pick and place robotics with the variation that object placement is dependent on the location of previously placed objects. Existing solutions in this domain typically consist of two primary components: a computer vision system to detect and localise the object of interest as well as a robot to alter the location and pose of the object in 3D space \cite{Sharath:Gantry_Robot_Design}.

A robot can be viewed as the combination of two core components, namely the robotic manipulator and the end-effector. The end-effector is the physical interface between the robot and the object of interest and is referred to as a robot gripper when its purpose is to grip the object to facilitate positional manipulation. The nature of the robot gripper depends on the physical characteristics of the object of interest and as such a wide variety of grippers have been developed. These include stiff finger grippers, flexible finger grippers, magnetic grippers and vacuum grippers which are best suited for objects with a flat surface \cite{Lundstrom:Industrial_Robot_Grippers}. The function of the robotic manipulator is to alter the position and pose of the end-effector in 3D space. Robotic manipulators are categorised by the coordinate systems used to describe their movement mechanics which includes polar, cylindrical, articulate and Cartesian coordinates \cite{Miller:Robots_and_Robotics_Principles}. Cartesian robots have the benefit that accuracy of the robot is uniform throughout the robot's work envelope.

The purpose of the computer vision system is to detect the object of interest and localise it using the input image data captured from the robot's workspace, so that the robot has sufficient information to interact with the object. This work is concerned with object detection as opposed to object recognition as the object of interest is known to be a cube. Computer vision techniques are used extract information from the image data in the form of features which are subsequently used to detect the objects of interest \cite{Kumar:Visual_Servoing}. An approach has been developed to detect generic rectangular cuboid objects in everyday scenes captured from a single perspective \cite{Xiao:Localizing_3D_Cuboids}. However, the generic nature of this task requires a highly sophisticated approach to achieve reasonable success. By rather constraining the scope of the detection problem to only the construction cubes within a controlled scene, far more accessible methods may be used. 

The contour of an object is a useful low-level feature that facilitates its detection. There are many approaches to contour detection algorithms such as the pixel-based, edge-based and region-based methodologies \cite{Gong2018:Overview_of_Contour_Detection_Approaches}. Contour detection, in conjunction with contour template matching, has been successfully used in robotic object detection and grasping applications using a single monocular camera \cite{Wei:Robotic_Object_Recognition_With_Natural_Background}. For objects with straight edges, the Hough transform is a useful image processing tool that can be used to capture these edges with parameterised straight lines in 2D space which is useful to determine the orientation of the object \cite{Aggarwal:Line_Detection_Hough_Transform}. The Canny edge detector is another popular algorithm that is useful for extracting low-level feature information from images in the form of edges \cite{Canny:Computational_Edge_Detection}. Lastly, in order to successfully localise an object, the points of the image coordinate system need to be mapped to the world coordinate system. This mapping requires the use of intrinsic and extrinsic parameters which describe the physical characteristics of the camera as well as its location and orientation respectively \cite{Szeliski:Computer_Vision_Algorithms_and_Applications}.

Overall, there has been a significant amount of research into artificial systems, consisting of a computer vision system used in conjunction with a robot to perform tasks such as 2D puzzle building or generic pick and place operations. However, the specific task of constructing moderately complex 3D shapes using small cubes does not appear to have been explored. Furthermore, a significant portion of research thus far has involved the use of an off-the-shelf robotic system. This work aims to develop a robot in conjunction with a computer vision system, based on a combination of image processing techniques discussed above specifically to solve the 3D shape construction task.

%% End of File.

