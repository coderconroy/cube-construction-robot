%%
%%  Department of Electrical, Electronic and Computer Engineering.
%%  EPR400/2 Final Report - Section 2.
%%  Copyright (C) 2011-2021 University of Pretoria.
%%

\section{Approach}

\subsection{Problem Space}

The nature of the selected approach to the cube construction task explored in this project is completely dependent on the characteristics of the problem space. The problem space was broadly defined as part of the project proposal. However, further details regarding the cube component of the problem space are required in order to justify the chosen approach. A number of general materials were considered to form the construction cubes including hard plastic, wood, aluminium and steel. Hard plastic cubes have the advantage that they are widely available off-the-shelf while wood offers ease in cube manufacturing. However, the low density of these materials means the cubes are more likely to shift in the shape structure when exposed to vibrations. Therefore, aluminium cubes were chosen due to their greater density. Aluminium was selected over steel due to its superior machinability and inability to rust.

\subsection{Robotic Subsystem}

The robotic subsystem (FU3) was identified as one of the major components of the solution system from a functional perspective in the project proposal. The high-level purpose of FU3 is to facilitate the manipulation of the construction cube's pose in 3D space. The robotic end-effector (FU3.5) acts as the robot's physical interface with the cube. Grippers are commonly used for the end-effector components. However, there exist planar arrangements of adjacent cubes that prevent access to at least one opposite face pair of the target cube which is required by the gripper to exert a grip. Therefore, a vacuum-based suction cup end-effector mechanism was selected as it only requires access to the top face of the target cube to exert a grip. Furthermore, the non-porous and smooth nature of the aluminum cubes render the cube amenable to this mechanism.

 There are a wide range of approaches to the robotic manipulator component (FU3.4) which is required to manipulate the end-effector pose in 3D space. These include the articulated robot, selective compliance articulated robot arm (SCARA), delta robot and the Cartesian robot. The articulated robot offers the greatest flexibility in the range of poses the robot can attain while the delta robot offers excellent movement speed. However, these kinematics of these robots are complex and minor imperfections in their implementation creates inaccuracy. Furthermore the precision of these robots, in addition to SCARA robots, varies throughout the workspace. Cartesian robots, on the other hand, exhibit consistently high precision throughout the workspace and are suited to Cartesian-based problems. A Cartesian gantry robot was selected as the robotic manipulator approach for these reasons. The robot was designed outwards from its interface with the cube in an iterative fashion  using computer-aided design (CAD) software and a mathematical base in dynamics.
 
 The robotic controller (FU3.2) was designed as an embedded platform to provide an interface to control the robotic manipulator and end-effector. This component was first designed and prototyped using a breadboard before a more robust PCB implementation was developed. The embedded software for this controller was prototyped using the STM32 hardware abstraction layer (HAL) library before being converted to first principles. The power supply (FU3.6) was taken off-the-shelf as well as the motor drivers (FU3.3). Finally, the communication unit (FU3.1) was based on the universal asynchronous receiver-transmitter (UART) and realised as a custom communication protocol used in conjunction with an off-the-shelf CH340 serial converter integrated circuit (IC).

\subsection{PC-Based Software Component}

The PC-based software component (FU2) was developed as a graphical user interface (GUI) application using C++ and the QT framework. C++ was selected due to the suitability of its performance to the computationally intensive nature of the image processing required in this project. QT was selected due to its maturity and support for OpenGL integration. FU2 was explored and developed in a bottom up approach. This means that the shape definition component (FU2.1) and cube detection and localisation component (FU2.2) were designed and developed first followed by the system controller (FU2.3), construction planner (FU2.4) and robotic motion planner (FU2.5) which all depend on FU2.1 and FU2.2. OpenGL was selected as the low-level graphics API to implement the 3D shape model render required by the shape definition component. This component was originally developed externally to the QT ecosystem to verify its functionality before being integrated adapted for integration with the QT OpenGL interface.

A number of object detection approaches were investigated for cube detection in FU2.2. However, a binary thresholding approach based on the reflection intensity of light from the top cube faces was found to be the most robust and used in conjunction with contour detection. A pin-hole camera model based approach was used to map points between the image and world frames for the purpose of cube localisation in FU2.2. These components were initially developed using the OpenCV library before being progressively replaced with first principle implementations.

Finally the system controller was developed simultaneously with the construction planner and the robotic motion planner in a tightly integrated manner. As the system controller acts as the central point that where the information pathways from the shape definition unit, cube detection and localisation unit and the robotic subsystem intersect, the system controller was developed as the top-level component in the PC-based software that integrates the other software components. As a result, the primary high-level software design work took place as part of the development of this component.

The approach reasoning and considerations discussed here correspond directly to the structure of the system design presented in Section \ref{sec:Design and Implementation}. Specifically the design of the robotic subsystem is first presented in two parts, namely the mechanical design in Section \ref{sec:Mechancial Robotic Component} followed by the embedded controller design in Section \ref{sec:Embedded Robot Controller}. The design of the PC-based software is then presented in three parts. The first two parts are the shape definition component in \ref{sec:Shape Definition Interface} and the computer vision component in Section \ref{sec:Computer Vision System} on which the system controller depends. Finally the system controller and the use of this component as the basis for system integration is presented in Section \ref{sec:System Controller}.


\newpage

%% End of File.

