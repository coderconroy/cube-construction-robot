%%
%%  Department of Electrical, Electronic and Computer Engineering.
%%  EPR400/2 Final Report - Section 2.
%%  Copyright (C) 2011-2021 University of Pretoria.
%%

\section{Approach}

\subsection{Problem Space}

The nature of the selected approach to the cube construction task explored in this project is completely dependent on the characteristics of the problem space. The problem space was broadly defined as part of the project proposal. However, further details regarding the cube component of the problem space are required in order to justify the chosen approach. A number of general materials were considered to form the construction cubes including hard plastic, wood, aluminium and steel. Hard plastic cubes have the advantage that they are widely available off-the-shelf while wood offers ease in cube manufacturing. However, the low density of these materials means the cubes are more likely to shift in the shape structure when exposed to vibrations. Therefore, aluminium cubes were chosen due to their greater density. Aluminium was selected over steel due to its superior machinability and inability to rust.

\subsection{Robotic Subsystem}

The robotic subsystem (FU3) was identified as one of the major components of the solution system from a functional perspective in the project proposal. The high-level purpose of FU3 is to facilitate the manipulation of the construction cube's pose in 3D space. The robotic end-effector (FU3.5) acts as the robot's physical interface with the cube. Grippers are commonly used for the end-effector components. However, there exist planar arrangements of adjacent cubes that prevent access to at least one opposite face pair of the target cube which is required by the gripper to exert a grip. Therefore, a vacuum-based suction cup end-effector mechanism was selected as it only requires access to the top face of the target cube to exert a grip. Furthermore, the non-porous and smooth nature of the aluminum cubes render the cube amenable to this mechanism.

 There are a wide range of approaches to the robotic manipulator component (FU3.4) which is required to manipulate the end-effector pose in 3D space. These include the articulated robot, selective compliance articulated robot arm (SCARA), delta robot and the Cartesian robot. The articulated robot offers the greatest flexibility in the range of poses the robot can attain while the delta robot offers excellent movement speed. However, these kinematics of these robots are complex and minor imperfections in their implementation creates inaccuracy. Furthermore the precision of these robots, in addition to SCARA robots, varies throughout the workspace. Cartesian robots, on the other hand, exhibit consistently high precision throughout the workspace and are suited to Cartesian-based problems. A Cartesian gantry robot was selected as the robotic manipulator approach for these reasons.


\subsection{PC-Based Software Component}

\ldots


\newpage

%% End of File.

