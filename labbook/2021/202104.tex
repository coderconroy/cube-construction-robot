\chapter[2021 April]{April 2021}

\section[2021/04/05]{Monday, 5 April 2021}

\subsection{Project Concept Overview}

The following key points were extracted from the project concept note for Mr Grobler's HG2 3D cube-world construction robot project:

\begin{compactitem}
    \item The project requires the manipulation of small cubes for which the quantity, material and dimensions are unspecified.
    \item The project requires the cubes be manipulated to construct various novel and moderately complex 3D shapes.
    \item The project requires a \ac{GUI} to be implemented to facilitate the definition of the 3D shapes to be constructed.
    \item The project requires that a robot is used to manipulate the cubes.
    \item The project requires the implementation of a vision system to allow the localisation of the robot and detection of the cubes.
\end{compactitem}

Based on the above points, the solution system required for the project can be subdivided into the following three broad subsystems:

\begin{compactitem}
    \item Robotic manipulator subsystem - The robotic mechanism required to manipulate the small cubes in order to alter their location and orientation in 3D space.
    \item Vision subsystem - The system required to localise the robotic manipulator and detect the position of the small cubes in 3D space.
    \item \ac{GUI} subsystem - The interface with a user that facilitates the definition of the 3D shape to be constructed by the robotic manipulator.
\end{compactitem}

\subsection{Initial Assumptions}

The project concept note leaves many details of the exact problem to be solved unspecified. In order to help focus the research for the project, the following assumptions about the cubes:

\begin{compactitem}
    \item All the cubes to be manipulated will have the same dimensions and consist of the same material.
    \item The dimensions of each cube will be approximately 10mm x 10mm x 10mm. This is based on the size of cubes in previous iterations of this project as discussed in the group meeting with Mr Grobler on 01/04/2021.
\end{compactitem}

\subsection{Literature Review: Robots and Robotics \cite{miller_miller_2017}}

This book proved useful in providing a high-level overview of the field of robotics and the terminology used within the field. The primary purpose of robots is to complete various tasks through the manipulation of parts, tools and materials. They are usually controlled by microprocessors which allows them to be programmed to fulfil specific functions which they can fulfil more accurately, efficiently and consistently than humans. The robot that needs to be developed for this project has similar functional needs to those of industrial robots. Industrial robots frequently make use of grippers to manipulate objects.

The three basic parts of a robot are the manipulator, controller and power source. The different types of robotic manipulators are distinguished by arm movements made by the manipulator which are classified by the coordinate systems used to describe them. These four coordinate systems are:

\begin{compactitem}
    \item Polar coordinates
    \item Cylindrical coordinates
    \item Cartesian coordinates
    \item Articulate coordinates
\end{compactitem}

The following terminology is common in the field of robotics and will be used going forward in this project unless otherwise stated:

\begin{compactitem}
    \item Base - The anchor point of the robot that supports all of its other components
    \item Arm - Component found on most robots that may contain joints and connects the base to the wrist
    \item Wrist - Component that connects the arm to the end effector and may support a wide range of motions
    \item Gripper / End effector - Component that holds the object to be manipulated and is attached at the end of the wrist. This component may also be a tool.
    \item Work envelope - The region in space where the robot is able to access with its end effector and perform it's task.
\end{compactitem}

The three primary drive mechanisms used for robotic manipulators are:

\begin{compactitem}
    \item Pneumatic drives
    \item Hydraulic drives
    \item Electric drives
\end{compactitem}

\pendsign

\section[2021/04/08]{Thursday, 8 April 2021}

\subsection{Individual Meeting}

An individual meeting took place with Mr Grobler from 15:00 until 15:25. The following were the important points to arise from the meeting:

\begin{compactitem}
    \item Cubes and a vacuum mechanism should be acquired as soon as possible in order to prototype the vacuum gripper and machince vision systems.
    \item Precision control of the robotic manipulator is one of the primary challenges of the practical.
    \item Oscillations and vibrations are two challenges with the precision placement of the end-effector and would likely be compounded if metallic cubes are used to to their extra mass.
    \item The localisation of the a cube entails figuring out its position in 3D space. The challenge arises when one tries to align the robotic manipulator with the cube in 3D space.
    \item A suggestion for the machine vision prototying is to take photos of a cube on various surfaces at varying angles and develop prototype code to identify the cubes in the photos.
    \item Metallic cubes may introduce difficulties into the system. The vacuum mechanism should be prototyped as soon as possible to see if it is capable of lifting a metallic cube or cubes of other materials.
    \item Unity is likely an overpowered solution for the \ac{GUI} and much more research is required for this component.
    \item A working prototype for all the subsystems or components thereof should be the target state of progress by the end of first semester.
\end{compactitem}

\pendsign

\section[2021/04/09]{Friday, 9 April 2021}

\subsection{Literature Review: Industrial Robot Grippers \cite{lundstrom_1974}}

This article outlines designs for many different robotic gripper variations according to the object or tool to be manipulated as well as other factors shuch as environmental conditions. The different robotic gripper categories are:

\begin{compactitem}
    \item Grippers with stiff fingers.
    \item Grippers with flexible fingers.
    \item Vacuum grippers.
    \item Magnetic grippers.
    \item Grippers with sensors.
    \item Miscellaneous grippers.
\end{compactitem}

Finger based grippers generally require at least two surfaces to be able to grip a given object while Vacuum grippers involve the use of rubber or plastic suction cups and offer several advantages over clamping grippers. Vacuum grippers are light and simple to build. The suction cup can be attached to a fitting which is attached to a spherical mounting. The tubes can be routed through a hollow arm. Piston, centrifugal and jet vacuum pumps can be used to generate the vacuum. Vacuum grippers are suited for objects with flat surfaces. Electromagnetic grippers are similar in that they can grip objects on a single plane. However, electromagnetic grippers can generate a larger attraction force per unit area, can grip faster and last longer than vacuum grippers. The other listed grippers do not appear to have immediate applicability to the project.

For objects that require a soft landing, springs can be used to attach the gripper mechanism to the robot wrist to ensure the gripper lands softly on the object. This may be beneficial for use with the cubes as a the effector could target a slightly lower Z position to ensure the cube is placed on a surfaces before release to ensure it doesn't drop and shift. A pneumatic activated counter balance can be used in conjunction with the spring to tune the soft landing. A cylindrical spigot joint can be used as an attachment mechanism to allow the gripper to rotate 270 degrees. The cubes will need to be rotated about the Z axis an this mechanism could potentially be used to enable this manipulation.

In the context of this project, a vacuum gripper appears the most promising for further investigation and prototyping for the following reasons:

\begin{compactitem}
    \item If the cube has two other cubes adjacent to it in the XY plane that are more difficult to move, such the the arrangement is similar to the corner of a square shape of height 1, a clamping gripper design is not able to grip the cube as it is unable to access two opposite faces from the Z direction. A vacuum gripper and electromagnetic gripper are able to manipulate the cube using a single face.
    \item A vacuum gripper is preferred over an electromagnetic gripper as it is not specified that the cubes consist of a ferromagnetic material which is an operating requirement of the electromagnetic gripper but not the vacuum gripper.
    \item The faces of cubes are flat which is suited to manipulation by a vacuum gripper.
\end{compactitem}

The choice to investigate the vacuum gripper further is not a final design choice and may change. The choice facilitates a plan of action for prototyping to assist in deciding the feasibility of system requirements.

\pendsign